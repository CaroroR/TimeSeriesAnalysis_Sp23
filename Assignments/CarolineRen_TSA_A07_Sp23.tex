% Options for packages loaded elsewhere
\PassOptionsToPackage{unicode}{hyperref}
\PassOptionsToPackage{hyphens}{url}
%
\documentclass[
]{article}
\usepackage{amsmath,amssymb}
\usepackage{lmodern}
\usepackage{iftex}
\ifPDFTeX
  \usepackage[T1]{fontenc}
  \usepackage[utf8]{inputenc}
  \usepackage{textcomp} % provide euro and other symbols
\else % if luatex or xetex
  \usepackage{unicode-math}
  \defaultfontfeatures{Scale=MatchLowercase}
  \defaultfontfeatures[\rmfamily]{Ligatures=TeX,Scale=1}
\fi
% Use upquote if available, for straight quotes in verbatim environments
\IfFileExists{upquote.sty}{\usepackage{upquote}}{}
\IfFileExists{microtype.sty}{% use microtype if available
  \usepackage[]{microtype}
  \UseMicrotypeSet[protrusion]{basicmath} % disable protrusion for tt fonts
}{}
\makeatletter
\@ifundefined{KOMAClassName}{% if non-KOMA class
  \IfFileExists{parskip.sty}{%
    \usepackage{parskip}
  }{% else
    \setlength{\parindent}{0pt}
    \setlength{\parskip}{6pt plus 2pt minus 1pt}}
}{% if KOMA class
  \KOMAoptions{parskip=half}}
\makeatother
\usepackage{xcolor}
\usepackage[margin=2.54cm]{geometry}
\usepackage{color}
\usepackage{fancyvrb}
\newcommand{\VerbBar}{|}
\newcommand{\VERB}{\Verb[commandchars=\\\{\}]}
\DefineVerbatimEnvironment{Highlighting}{Verbatim}{commandchars=\\\{\}}
% Add ',fontsize=\small' for more characters per line
\usepackage{framed}
\definecolor{shadecolor}{RGB}{248,248,248}
\newenvironment{Shaded}{\begin{snugshade}}{\end{snugshade}}
\newcommand{\AlertTok}[1]{\textcolor[rgb]{0.94,0.16,0.16}{#1}}
\newcommand{\AnnotationTok}[1]{\textcolor[rgb]{0.56,0.35,0.01}{\textbf{\textit{#1}}}}
\newcommand{\AttributeTok}[1]{\textcolor[rgb]{0.77,0.63,0.00}{#1}}
\newcommand{\BaseNTok}[1]{\textcolor[rgb]{0.00,0.00,0.81}{#1}}
\newcommand{\BuiltInTok}[1]{#1}
\newcommand{\CharTok}[1]{\textcolor[rgb]{0.31,0.60,0.02}{#1}}
\newcommand{\CommentTok}[1]{\textcolor[rgb]{0.56,0.35,0.01}{\textit{#1}}}
\newcommand{\CommentVarTok}[1]{\textcolor[rgb]{0.56,0.35,0.01}{\textbf{\textit{#1}}}}
\newcommand{\ConstantTok}[1]{\textcolor[rgb]{0.00,0.00,0.00}{#1}}
\newcommand{\ControlFlowTok}[1]{\textcolor[rgb]{0.13,0.29,0.53}{\textbf{#1}}}
\newcommand{\DataTypeTok}[1]{\textcolor[rgb]{0.13,0.29,0.53}{#1}}
\newcommand{\DecValTok}[1]{\textcolor[rgb]{0.00,0.00,0.81}{#1}}
\newcommand{\DocumentationTok}[1]{\textcolor[rgb]{0.56,0.35,0.01}{\textbf{\textit{#1}}}}
\newcommand{\ErrorTok}[1]{\textcolor[rgb]{0.64,0.00,0.00}{\textbf{#1}}}
\newcommand{\ExtensionTok}[1]{#1}
\newcommand{\FloatTok}[1]{\textcolor[rgb]{0.00,0.00,0.81}{#1}}
\newcommand{\FunctionTok}[1]{\textcolor[rgb]{0.00,0.00,0.00}{#1}}
\newcommand{\ImportTok}[1]{#1}
\newcommand{\InformationTok}[1]{\textcolor[rgb]{0.56,0.35,0.01}{\textbf{\textit{#1}}}}
\newcommand{\KeywordTok}[1]{\textcolor[rgb]{0.13,0.29,0.53}{\textbf{#1}}}
\newcommand{\NormalTok}[1]{#1}
\newcommand{\OperatorTok}[1]{\textcolor[rgb]{0.81,0.36,0.00}{\textbf{#1}}}
\newcommand{\OtherTok}[1]{\textcolor[rgb]{0.56,0.35,0.01}{#1}}
\newcommand{\PreprocessorTok}[1]{\textcolor[rgb]{0.56,0.35,0.01}{\textit{#1}}}
\newcommand{\RegionMarkerTok}[1]{#1}
\newcommand{\SpecialCharTok}[1]{\textcolor[rgb]{0.00,0.00,0.00}{#1}}
\newcommand{\SpecialStringTok}[1]{\textcolor[rgb]{0.31,0.60,0.02}{#1}}
\newcommand{\StringTok}[1]{\textcolor[rgb]{0.31,0.60,0.02}{#1}}
\newcommand{\VariableTok}[1]{\textcolor[rgb]{0.00,0.00,0.00}{#1}}
\newcommand{\VerbatimStringTok}[1]{\textcolor[rgb]{0.31,0.60,0.02}{#1}}
\newcommand{\WarningTok}[1]{\textcolor[rgb]{0.56,0.35,0.01}{\textbf{\textit{#1}}}}
\usepackage{graphicx}
\makeatletter
\def\maxwidth{\ifdim\Gin@nat@width>\linewidth\linewidth\else\Gin@nat@width\fi}
\def\maxheight{\ifdim\Gin@nat@height>\textheight\textheight\else\Gin@nat@height\fi}
\makeatother
% Scale images if necessary, so that they will not overflow the page
% margins by default, and it is still possible to overwrite the defaults
% using explicit options in \includegraphics[width, height, ...]{}
\setkeys{Gin}{width=\maxwidth,height=\maxheight,keepaspectratio}
% Set default figure placement to htbp
\makeatletter
\def\fps@figure{htbp}
\makeatother
\setlength{\emergencystretch}{3em} % prevent overfull lines
\providecommand{\tightlist}{%
  \setlength{\itemsep}{0pt}\setlength{\parskip}{0pt}}
\setcounter{secnumdepth}{-\maxdimen} % remove section numbering
\ifLuaTeX
  \usepackage{selnolig}  % disable illegal ligatures
\fi
\IfFileExists{bookmark.sty}{\usepackage{bookmark}}{\usepackage{hyperref}}
\IfFileExists{xurl.sty}{\usepackage{xurl}}{} % add URL line breaks if available
\urlstyle{same} % disable monospaced font for URLs
\hypersetup{
  pdftitle={ENV 790.30 - Time Series Analysis for Energy Data \textbar{} Spring 2023},
  pdfauthor={Student Name},
  hidelinks,
  pdfcreator={LaTeX via pandoc}}

\title{ENV 790.30 - Time Series Analysis for Energy Data \textbar{}
Spring 2023}
\usepackage{etoolbox}
\makeatletter
\providecommand{\subtitle}[1]{% add subtitle to \maketitle
  \apptocmd{\@title}{\par {\large #1 \par}}{}{}
}
\makeatother
\subtitle{Assignment 7 - Due date 03/20/23}
\author{Student Name}
\date{}

\begin{document}
\maketitle

\hypertarget{directions}{%
\subsection{Directions}\label{directions}}

You should open the .rmd file corresponding to this assignment on
RStudio. The file is available on our class repository on Github. And to
do so you will need to fork our repository and link it to your RStudio.

Once you have the file open on your local machine the first thing you
will do is rename the file such that it includes your first and last
name (e.g., ``LuanaLima\_TSA\_A07\_Sp23.Rmd''). Then change ``Student
Name'' on line 4 with your name.

Then you will start working through the assignment by \textbf{creating
code and output} that answer each question. Be sure to use this
assignment document. Your report should contain the answer to each
question and any plots/tables you obtained (when applicable).

When you have completed the assignment, \textbf{Knit} the text and code
into a single PDF file. Submit this pdf using Sakai.

\hypertarget{set-up}{%
\subsection{Set up}\label{set-up}}

\begin{Shaded}
\begin{Highlighting}[]
\CommentTok{\#Load/install required package here}
\FunctionTok{library}\NormalTok{(lubridate)}
\end{Highlighting}
\end{Shaded}

\begin{verbatim}
## 
## 载入程辑包:'lubridate'
\end{verbatim}

\begin{verbatim}
## The following objects are masked from 'package:base':
## 
##     date, intersect, setdiff, union
\end{verbatim}

\begin{Shaded}
\begin{Highlighting}[]
\FunctionTok{library}\NormalTok{(ggplot2)}
\FunctionTok{library}\NormalTok{(forecast)  }
\end{Highlighting}
\end{Shaded}

\begin{verbatim}
## Registered S3 method overwritten by 'quantmod':
##   method            from
##   as.zoo.data.frame zoo
\end{verbatim}

\begin{Shaded}
\begin{Highlighting}[]
\FunctionTok{library}\NormalTok{(Kendall)}
\FunctionTok{library}\NormalTok{(tseries)}
\FunctionTok{library}\NormalTok{(outliers)}
\FunctionTok{library}\NormalTok{(tidyverse)}
\end{Highlighting}
\end{Shaded}

\begin{verbatim}
## -- Attaching packages --------------------------------------- tidyverse 1.3.2
## --
\end{verbatim}

\begin{verbatim}
## v tibble  3.1.8     v dplyr   1.1.0
## v tidyr   1.3.0     v stringr 1.5.0
## v readr   2.1.3     v forcats 1.0.0
## v purrr   1.0.1     
## -- Conflicts ------------------------------------------ tidyverse_conflicts() --
## x lubridate::as.difftime() masks base::as.difftime()
## x lubridate::date()        masks base::date()
## x dplyr::filter()          masks stats::filter()
## x lubridate::intersect()   masks base::intersect()
## x dplyr::lag()             masks stats::lag()
## x lubridate::setdiff()     masks base::setdiff()
## x lubridate::union()       masks base::union()
\end{verbatim}

\begin{Shaded}
\begin{Highlighting}[]
\FunctionTok{library}\NormalTok{(smooth)}
\end{Highlighting}
\end{Shaded}

\begin{verbatim}
## Warning: 程辑包'smooth'是用R版本4.2.3 来建造的
\end{verbatim}

\begin{verbatim}
## 载入需要的程辑包:greybox
\end{verbatim}

\begin{verbatim}
## Warning: 程辑包'greybox'是用R版本4.2.3 来建造的
\end{verbatim}

\begin{verbatim}
## Package "greybox", v1.0.7 loaded.
## 
## 
## 载入程辑包:'greybox'
## 
## The following object is masked from 'package:tidyr':
## 
##     spread
## 
## The following object is masked from 'package:lubridate':
## 
##     hm
## 
## This is package "smooth", v3.2.0
\end{verbatim}

\begin{Shaded}
\begin{Highlighting}[]
\FunctionTok{library}\NormalTok{(trend)}
\end{Highlighting}
\end{Shaded}

\begin{verbatim}
## Warning: 程辑包'trend'是用R版本4.2.3 来建造的
\end{verbatim}

\begin{Shaded}
\begin{Highlighting}[]
\FunctionTok{library}\NormalTok{(tseries)}
\FunctionTok{library}\NormalTok{(forecast)}
\end{Highlighting}
\end{Shaded}

\hypertarget{importing-and-processing-the-data-set}{%
\subsection{Importing and processing the data
set}\label{importing-and-processing-the-data-set}}

Consider the data from the file
``Net\_generation\_United\_States\_all\_sectors\_monthly.csv''. The data
corresponds to the monthly net generation from January 2001 to December
2020 by source and is provided by the US Energy Information and
Administration. \textbf{You will work with the natural gas column only}.

Packages needed for this assignment: ``forecast'',``tseries''. Do not
forget to load them before running your script, since they are NOT
default packages.\textbackslash{}

\hypertarget{q1}{%
\subsubsection{Q1}\label{q1}}

Import the csv file and create a time series object for natural gas.
Make you sure you specify the \textbf{start=} and \textbf{frequency=}
arguments. Plot the time series over time, ACF and PACF.

\begin{Shaded}
\begin{Highlighting}[]
\NormalTok{naturalgas\_data }\OtherTok{\textless{}{-}} \FunctionTok{read.csv}\NormalTok{(}
  \AttributeTok{file=}\StringTok{"../Data/Net\_generation\_United\_States\_all\_sectors\_monthly.csv"}\NormalTok{,}
  \AttributeTok{header=}\ConstantTok{TRUE}\NormalTok{,}\AttributeTok{skip=}\DecValTok{4}\NormalTok{)}

\NormalTok{natural\_gas\_only }\OtherTok{\textless{}{-}} \FunctionTok{ts}\NormalTok{(naturalgas\_data}\SpecialCharTok{$}\StringTok{\textasciigrave{}}\AttributeTok{natural.gas.thousand.megawatthours}\StringTok{\textasciigrave{}}\NormalTok{, }\AttributeTok{start=}\FunctionTok{c}\NormalTok{(}\DecValTok{2001}\NormalTok{,}\DecValTok{1}\NormalTok{), }\AttributeTok{frequency=}\DecValTok{12}\NormalTok{)}
\end{Highlighting}
\end{Shaded}

\begin{Shaded}
\begin{Highlighting}[]
\CommentTok{\# Time series plot}
\FunctionTok{plot}\NormalTok{(natural\_gas\_only, }\AttributeTok{main=}\StringTok{"Monthly Natural Gas Generation in the US"}\NormalTok{, }\AttributeTok{ylab=}\StringTok{"Billion Kilowatt{-}hours"}\NormalTok{)}
\end{Highlighting}
\end{Shaded}

\includegraphics{CarolineRen_TSA_A07_Sp23_files/figure-latex/unnamed-chunk-3-1.pdf}

\begin{Shaded}
\begin{Highlighting}[]
\CommentTok{\# ACF and PACF plots}
\FunctionTok{par}\NormalTok{(}\AttributeTok{mfrow=}\FunctionTok{c}\NormalTok{(}\DecValTok{1}\NormalTok{,}\DecValTok{2}\NormalTok{))}
\FunctionTok{acf}\NormalTok{(natural\_gas\_only, }\AttributeTok{main=}\StringTok{"ACF of Natural Gas Generation"}\NormalTok{)}
\FunctionTok{pacf}\NormalTok{(natural\_gas\_only, }\AttributeTok{main=}\StringTok{"PACF of Natural Gas Generation"}\NormalTok{)}
\end{Highlighting}
\end{Shaded}

\includegraphics{CarolineRen_TSA_A07_Sp23_files/figure-latex/unnamed-chunk-3-2.pdf}

\hypertarget{q2}{%
\subsubsection{Q2}\label{q2}}

Using the \(decompose()\) or \(stl()\) and the \(seasadj()\) functions
create a series without the seasonal component, i.e., a deseasonalized
natural gas series. Plot the deseasonalized series over time and
corresponding ACF and PACF. Compare with the plots obtained in Q1.

\begin{Shaded}
\begin{Highlighting}[]
\CommentTok{\# Decompose the time series}
\NormalTok{decomp\_ng }\OtherTok{\textless{}{-}} \FunctionTok{decompose}\NormalTok{(natural\_gas\_only)}

\CommentTok{\# Extract the seasonally adjusted component}
\NormalTok{seas\_adj\_ng }\OtherTok{\textless{}{-}} \FunctionTok{seasadj}\NormalTok{(decomp\_ng)}

\CommentTok{\# Plot the deseasonalized series}
\FunctionTok{plot}\NormalTok{(seas\_adj\_ng, }\AttributeTok{main=}\StringTok{"Monthly Natural Gas Generation in the US (Seasonally Adjusted)"}\NormalTok{, }\AttributeTok{ylab=}\StringTok{"Billion Kilowatt{-}hours"}\NormalTok{)}
\end{Highlighting}
\end{Shaded}

\includegraphics{CarolineRen_TSA_A07_Sp23_files/figure-latex/unnamed-chunk-4-1.pdf}

\begin{Shaded}
\begin{Highlighting}[]
\CommentTok{\# ACF and PACF plots of the deseasonalized series}
\FunctionTok{par}\NormalTok{(}\AttributeTok{mfrow=}\FunctionTok{c}\NormalTok{(}\DecValTok{1}\NormalTok{,}\DecValTok{2}\NormalTok{))}
\FunctionTok{acf}\NormalTok{(seas\_adj\_ng, }\AttributeTok{main=}\StringTok{"ACF of Natural Gas Generation (Seasonally Adjusted)"}\NormalTok{)}
\FunctionTok{pacf}\NormalTok{(seas\_adj\_ng, }\AttributeTok{main=}\StringTok{"PACF of Natural Gas Generation (Seasonally Adjusted)"}\NormalTok{)}
\end{Highlighting}
\end{Shaded}

\includegraphics{CarolineRen_TSA_A07_Sp23_files/figure-latex/unnamed-chunk-4-2.pdf}

\hypertarget{modeling-the-seasonally-adjusted-or-deseasonalized-series}{%
\subsection{Modeling the seasonally adjusted or deseasonalized
series}\label{modeling-the-seasonally-adjusted-or-deseasonalized-series}}

\hypertarget{q3}{%
\subsubsection{Q3}\label{q3}}

Run the ADF test and Mann Kendall test on the deseasonalized data from
Q2. Report and explain the results.

\begin{Shaded}
\begin{Highlighting}[]
\CommentTok{\# ADF test}
\NormalTok{adf\_test }\OtherTok{\textless{}{-}} \FunctionTok{adf.test}\NormalTok{(seas\_adj\_ng)}
\end{Highlighting}
\end{Shaded}

\begin{verbatim}
## Warning in adf.test(seas_adj_ng): p-value smaller than printed p-value
\end{verbatim}

\begin{Shaded}
\begin{Highlighting}[]
\FunctionTok{cat}\NormalTok{(}\StringTok{"ADF test p{-}value:"}\NormalTok{, adf\_test}\SpecialCharTok{$}\NormalTok{p.value, }\StringTok{"}\SpecialCharTok{\textbackslash{}n}\StringTok{"}\NormalTok{)}
\end{Highlighting}
\end{Shaded}

\begin{verbatim}
## ADF test p-value: 0.01
\end{verbatim}

\begin{Shaded}
\begin{Highlighting}[]
\ControlFlowTok{if}\NormalTok{(adf\_test}\SpecialCharTok{$}\NormalTok{p.value }\SpecialCharTok{\textless{}} \FloatTok{0.05}\NormalTok{)\{}
  \FunctionTok{cat}\NormalTok{(}\StringTok{"The deseasonalized natural gas time series is stationary.}\SpecialCharTok{\textbackslash{}n}\StringTok{"}\NormalTok{)}
\NormalTok{\}}\ControlFlowTok{else}\NormalTok{\{}
  \FunctionTok{cat}\NormalTok{(}\StringTok{"The deseasonalized natural gas time series is non{-}stationary.}\SpecialCharTok{\textbackslash{}n}\StringTok{"}\NormalTok{)}
\NormalTok{\}}
\end{Highlighting}
\end{Shaded}

\begin{verbatim}
## The deseasonalized natural gas time series is stationary.
\end{verbatim}

\begin{Shaded}
\begin{Highlighting}[]
\CommentTok{\# Mann{-}Kendall test}
\NormalTok{mk\_test }\OtherTok{\textless{}{-}} \FunctionTok{mk.test}\NormalTok{(seas\_adj\_ng)}
\FunctionTok{cat}\NormalTok{(}\StringTok{"Mann{-}Kendall test p{-}value:"}\NormalTok{, mk\_test}\SpecialCharTok{$}\NormalTok{p.value, }\StringTok{"}\SpecialCharTok{\textbackslash{}n}\StringTok{"}\NormalTok{)}
\end{Highlighting}
\end{Shaded}

\begin{verbatim}
## Mann-Kendall test p-value: 2.700333e-84
\end{verbatim}

\begin{Shaded}
\begin{Highlighting}[]
\ControlFlowTok{if}\NormalTok{(mk\_test}\SpecialCharTok{$}\NormalTok{p.value }\SpecialCharTok{\textless{}} \FloatTok{0.05}\NormalTok{)\{}
  \FunctionTok{cat}\NormalTok{(}\StringTok{"There is a significant trend in the deseasonalized natural gas time series.}\SpecialCharTok{\textbackslash{}n}\StringTok{"}\NormalTok{)}
\NormalTok{\}}\ControlFlowTok{else}\NormalTok{\{}
  \FunctionTok{cat}\NormalTok{(}\StringTok{"There is no significant trend in the deseasonalized natural gas time series.}\SpecialCharTok{\textbackslash{}n}\StringTok{"}\NormalTok{)}
\NormalTok{\}}
\end{Highlighting}
\end{Shaded}

\begin{verbatim}
## There is a significant trend in the deseasonalized natural gas time series.
\end{verbatim}

\hypertarget{q4}{%
\subsubsection{Q4}\label{q4}}

Using the plots from Q2 and test results from Q3 identify the ARIMA
model parameters \(p,d\) and \(q\). Note that in this case because you
removed the seasonal component prior to identifying the model you don't
need to worry about seasonal component. Clearly state your criteria and
any additional function in R you might use. DO NOT use the
\(auto.arima()\) function. You will be evaluated on ability to can read
the plots and interpret the test results.

\begin{Shaded}
\begin{Highlighting}[]
\CommentTok{\# Fit an ARIMA model to the deseasonalized series}
\NormalTok{fit }\OtherTok{\textless{}{-}} \FunctionTok{Arima}\NormalTok{(seas\_adj\_ng, }\AttributeTok{order=}\FunctionTok{c}\NormalTok{(}\DecValTok{1}\NormalTok{,}\DecValTok{0}\NormalTok{,}\DecValTok{0}\NormalTok{))}

\CommentTok{\# Use the auto.arima() function to find the best{-}fitting model}
\NormalTok{fit\_auto }\OtherTok{\textless{}{-}} \FunctionTok{auto.arima}\NormalTok{(seas\_adj\_ng)}

\CommentTok{\# Compare the two models}
\FunctionTok{summary}\NormalTok{(fit)}
\end{Highlighting}
\end{Shaded}

\begin{verbatim}
## Series: seas_adj_ng 
## ARIMA(1,0,0) with non-zero mean 
## 
## Coefficients:
##          ar1      mean
##       0.9827  88932.70
## s.e.  0.0121  17013.58
## 
## sigma^2 = 30849844:  log likelihood = -2410.58
## AIC=4827.17   AICc=4827.27   BIC=4837.61
## 
## Training set error measures:
##                     ME     RMSE      MAE        MPE     MAPE      MASE
## Training set -331.9674 5531.072 4331.651 -0.8041255 5.309321 0.5295791
##                    ACF1
## Training set -0.1501136
\end{verbatim}

\begin{Shaded}
\begin{Highlighting}[]
\FunctionTok{summary}\NormalTok{(fit\_auto)}
\end{Highlighting}
\end{Shaded}

\begin{verbatim}
## Series: seas_adj_ng 
## ARIMA(3,1,0)(1,0,1)[12] with drift 
## 
## Coefficients:
##           ar1      ar2      ar3    sar1     sma1      drift
##       -0.2028  -0.1851  -0.1378  0.6609  -0.4698  -331.8138
## s.e.   0.0645   0.0655   0.0682  0.1918   0.2120   328.8248
## 
## sigma^2 = 27791547:  log likelihood = -2384.89
## AIC=4783.79   AICc=4784.27   BIC=4808.12
## 
## Training set error measures:
##                     ME    RMSE      MAE        MPE     MAPE      MASE
## Training set -16.33675 5194.32 4013.071 -0.2787641 4.930523 0.4906302
##                     ACF1
## Training set -0.01784376
\end{verbatim}

\hypertarget{q5}{%
\subsubsection{Q5}\label{q5}}

Use \(Arima()\) from package ``forecast'' to fit an ARIMA model to your
series considering the order estimated in Q4. You should allow constants
in the model, i.e., \(include.mean = TRUE\) or \(include.drift=TRUE\).
\textbf{Print the coefficients} in your report. Hint: use the \(cat()\)
function to print.

\begin{Shaded}
\begin{Highlighting}[]
\CommentTok{\# Fit the ARIMA model}
\NormalTok{model }\OtherTok{\textless{}{-}} \FunctionTok{Arima}\NormalTok{(seas\_adj\_ng, }\AttributeTok{order=}\FunctionTok{c}\NormalTok{(}\DecValTok{2}\NormalTok{,}\DecValTok{1}\NormalTok{,}\DecValTok{1}\NormalTok{), }\AttributeTok{include.mean=}\ConstantTok{TRUE}\NormalTok{)}

\CommentTok{\# Print the coefficients}
\FunctionTok{cat}\NormalTok{(}\StringTok{"Coefficients:"}\NormalTok{)}
\end{Highlighting}
\end{Shaded}

\begin{verbatim}
## Coefficients:
\end{verbatim}

\begin{Shaded}
\begin{Highlighting}[]
\FunctionTok{coef}\NormalTok{(model)}
\end{Highlighting}
\end{Shaded}

\begin{verbatim}
##        ar1        ar2        ma1 
##  0.5564536 -0.0689051 -0.7731045
\end{verbatim}

\hypertarget{q6}{%
\subsubsection{Q6}\label{q6}}

Now plot the residuals of the ARIMA fit from Q5 along with residuals ACF
and PACF on the same window. You may use the \(checkresiduals()\)
function to automatically generate the three plots. Do the residual
series look like a white noise series? Why?

\begin{Shaded}
\begin{Highlighting}[]
\NormalTok{fit\_arima }\OtherTok{\textless{}{-}} \FunctionTok{Arima}\NormalTok{(seas\_adj\_ng, }\AttributeTok{order=}\FunctionTok{c}\NormalTok{(}\DecValTok{0}\NormalTok{, }\DecValTok{1}\NormalTok{, }\DecValTok{2}\NormalTok{), }\AttributeTok{include.mean=}\ConstantTok{TRUE}\NormalTok{)}
\FunctionTok{checkresiduals}\NormalTok{(fit\_arima)}
\end{Highlighting}
\end{Shaded}

\includegraphics{CarolineRen_TSA_A07_Sp23_files/figure-latex/unnamed-chunk-8-1.pdf}

\begin{verbatim}
## 
##  Ljung-Box test
## 
## data:  Residuals from ARIMA(0,1,2)
## Q* = 59.834, df = 22, p-value = 2.365e-05
## 
## Model df: 2.   Total lags used: 24
\end{verbatim}

We have residuals from the arima model centered around the zero, which
is a good sign of model fittness. (We don't want them to be bias, or
distributed unevenly). In addition, both ACF and PACF have no
significant spikes beyond the signifance bounds. Our residuals are
distributed similar to white noise process, meaning that we have account
all the factors in the model.

\hypertarget{modeling-the-original-series-with-seasonality}{%
\subsection{Modeling the original series (with
seasonality)}\label{modeling-the-original-series-with-seasonality}}

\hypertarget{q7}{%
\subsubsection{Q7}\label{q7}}

Repeat Q4-Q6 for the original series (the complete series that has the
seasonal component). Note that when you model the seasonal series, you
need to specify the seasonal part of the ARIMA model as well, i.e.,
\(P\), \(D\) and \(Q\).

\begin{Shaded}
\begin{Highlighting}[]
\FunctionTok{par}\NormalTok{(}\AttributeTok{mfrow=}\FunctionTok{c}\NormalTok{(}\DecValTok{1}\NormalTok{,}\DecValTok{2}\NormalTok{))}
\FunctionTok{acf}\NormalTok{(natural\_gas\_only, }\AttributeTok{main=}\StringTok{"ACF of Natural Gas Generation (Original Series)"}\NormalTok{)}
\FunctionTok{pacf}\NormalTok{(natural\_gas\_only, }\AttributeTok{main=}\StringTok{"PACF of Natural Gas Generation (Original Series)"}\NormalTok{)}
\end{Highlighting}
\end{Shaded}

\includegraphics{CarolineRen_TSA_A07_Sp23_files/figure-latex/unnamed-chunk-9-1.pdf}

\begin{Shaded}
\begin{Highlighting}[]
\FunctionTok{adf.test}\NormalTok{(natural\_gas\_only)}
\end{Highlighting}
\end{Shaded}

\begin{verbatim}
## Warning in adf.test(natural_gas_only): p-value smaller than printed p-value
\end{verbatim}

\begin{verbatim}
## 
##  Augmented Dickey-Fuller Test
## 
## data:  natural_gas_only
## Dickey-Fuller = -8.8795, Lag order = 6, p-value = 0.01
## alternative hypothesis: stationary
\end{verbatim}

\begin{Shaded}
\begin{Highlighting}[]
\FunctionTok{MannKendall}\NormalTok{(natural\_gas\_only)}
\end{Highlighting}
\end{Shaded}

\begin{verbatim}
## tau = -0.651, 2-sided pvalue =< 2.22e-16
\end{verbatim}

\begin{Shaded}
\begin{Highlighting}[]
\NormalTok{nat\_gas\_arima }\OtherTok{\textless{}{-}} \FunctionTok{Arima}\NormalTok{(natural\_gas\_only, }\AttributeTok{order=}\FunctionTok{c}\NormalTok{(}\DecValTok{1}\NormalTok{,}\DecValTok{1}\NormalTok{,}\DecValTok{1}\NormalTok{), }\AttributeTok{include.mean=}\ConstantTok{TRUE}\NormalTok{)}
\FunctionTok{cat}\NormalTok{(}\StringTok{"Coefficients:"}\NormalTok{)}
\end{Highlighting}
\end{Shaded}

\begin{verbatim}
## Coefficients:
\end{verbatim}

\begin{Shaded}
\begin{Highlighting}[]
\FunctionTok{coef}\NormalTok{(nat\_gas\_arima)}
\end{Highlighting}
\end{Shaded}

\begin{verbatim}
##        ar1        ma1 
## 0.09984625 0.30759627
\end{verbatim}

\begin{Shaded}
\begin{Highlighting}[]
\FunctionTok{checkresiduals}\NormalTok{(nat\_gas\_arima)}
\end{Highlighting}
\end{Shaded}

\includegraphics{CarolineRen_TSA_A07_Sp23_files/figure-latex/unnamed-chunk-13-1.pdf}

\begin{verbatim}
## 
##  Ljung-Box test
## 
## data:  Residuals from ARIMA(1,1,1)
## Q* = 423.99, df = 22, p-value < 2.2e-16
## 
## Model df: 2.   Total lags used: 24
\end{verbatim}

\hypertarget{q8}{%
\subsubsection{Q8}\label{q8}}

Compare the residual series for Q7 and Q6. Can you tell which ARIMA
model is better representing the Natural Gas Series? Is that a fair
comparison? Explain your response.

The residuals for Q6 perform better than residuals for Q7. Residuals for
Q6 resembles white noise series, there is no visible pattern or bias,
meaning that we are successful at capturing all relevant factors.
Residuals for Q7 still show significant pikes in ACF and PACF plots, we
are likely missing some factors.

It is not fair to compare two models directly. Model for Q6 is focusing
on non seasonal part of the data only, while Q7 takes into account of
both seasonal and non-seasonal components. A bettter measure would be
AIC or BIC.

\hypertarget{checking-your-model-with-the-auto.arima}{%
\subsection{Checking your model with the
auto.arima()}\label{checking-your-model-with-the-auto.arima}}

\textbf{Please} do not change your answers for Q4 and Q7 after you ran
the \(auto.arima()\). It is \textbf{ok} if you didn't get all orders
correctly. You will not loose points for not having the same order as
the \(auto.arima()\).

\hypertarget{q9}{%
\subsubsection{Q9}\label{q9}}

Use the \(auto.arima()\) command on the \textbf{deseasonalized series}
to let R choose the model parameter for you. What's the order of the
best ARIMA model? Does it match what you specified in Q4?

\begin{Shaded}
\begin{Highlighting}[]
\CommentTok{\# Use auto.arima on the deseasonalized series}
\NormalTok{auto\_arima\_ng }\OtherTok{\textless{}{-}} \FunctionTok{auto.arima}\NormalTok{(seas\_adj\_ng)}

\CommentTok{\# Print the order of the best ARIMA model}
\FunctionTok{cat}\NormalTok{(}\StringTok{"Order of the best ARIMA model for deseasonalized series: "}\NormalTok{, auto\_arima\_ng}\SpecialCharTok{$}\NormalTok{arma[}\FunctionTok{c}\NormalTok{(}\DecValTok{1}\NormalTok{,}\DecValTok{3}\NormalTok{)], }\StringTok{"}\SpecialCharTok{\textbackslash{}n}\StringTok{"}\NormalTok{)}
\end{Highlighting}
\end{Shaded}

\begin{verbatim}
## Order of the best ARIMA model for deseasonalized series:  3 1
\end{verbatim}

\hypertarget{q10}{%
\subsubsection{Q10}\label{q10}}

Use the \(auto.arima()\) command on the \textbf{original series} to let
R choose the model parameters for you. Does it match what you specified
in Q7?

\begin{Shaded}
\begin{Highlighting}[]
\CommentTok{\# Fit ARIMA model using auto.arima}
\NormalTok{arima\_model }\OtherTok{\textless{}{-}} \FunctionTok{auto.arima}\NormalTok{(natural\_gas\_only)}
\NormalTok{arima\_model}
\end{Highlighting}
\end{Shaded}

\begin{verbatim}
## Series: natural_gas_only 
## ARIMA(2,0,1)(2,1,2)[12] with drift 
## 
## Coefficients:
##          ar1      ar2      ma1     sar1     sar2     sma1    sma2      drift
##       1.1650  -0.2834  -0.4837  -0.0667  -0.0785  -0.6371  0.0072  -357.8436
## s.e.  0.4163   0.3180   0.3992   1.3206   0.1014   1.3199  0.9204    44.1285
## 
## sigma^2 = 27958165:  log likelihood = -2278.46
## AIC=4574.91   AICc=4575.74   BIC=4605.77
\end{verbatim}

\end{document}
