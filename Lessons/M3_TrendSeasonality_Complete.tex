% Options for packages loaded elsewhere
\PassOptionsToPackage{unicode}{hyperref}
\PassOptionsToPackage{hyphens}{url}
%
\documentclass[
]{article}
\usepackage{amsmath,amssymb}
\usepackage{lmodern}
\usepackage{iftex}
\ifPDFTeX
  \usepackage[T1]{fontenc}
  \usepackage[utf8]{inputenc}
  \usepackage{textcomp} % provide euro and other symbols
\else % if luatex or xetex
  \usepackage{unicode-math}
  \defaultfontfeatures{Scale=MatchLowercase}
  \defaultfontfeatures[\rmfamily]{Ligatures=TeX,Scale=1}
\fi
% Use upquote if available, for straight quotes in verbatim environments
\IfFileExists{upquote.sty}{\usepackage{upquote}}{}
\IfFileExists{microtype.sty}{% use microtype if available
  \usepackage[]{microtype}
  \UseMicrotypeSet[protrusion]{basicmath} % disable protrusion for tt fonts
}{}
\makeatletter
\@ifundefined{KOMAClassName}{% if non-KOMA class
  \IfFileExists{parskip.sty}{%
    \usepackage{parskip}
  }{% else
    \setlength{\parindent}{0pt}
    \setlength{\parskip}{6pt plus 2pt minus 1pt}}
}{% if KOMA class
  \KOMAoptions{parskip=half}}
\makeatother
\usepackage{xcolor}
\usepackage[margin=1in]{geometry}
\usepackage{color}
\usepackage{fancyvrb}
\newcommand{\VerbBar}{|}
\newcommand{\VERB}{\Verb[commandchars=\\\{\}]}
\DefineVerbatimEnvironment{Highlighting}{Verbatim}{commandchars=\\\{\}}
% Add ',fontsize=\small' for more characters per line
\usepackage{framed}
\definecolor{shadecolor}{RGB}{248,248,248}
\newenvironment{Shaded}{\begin{snugshade}}{\end{snugshade}}
\newcommand{\AlertTok}[1]{\textcolor[rgb]{0.94,0.16,0.16}{#1}}
\newcommand{\AnnotationTok}[1]{\textcolor[rgb]{0.56,0.35,0.01}{\textbf{\textit{#1}}}}
\newcommand{\AttributeTok}[1]{\textcolor[rgb]{0.77,0.63,0.00}{#1}}
\newcommand{\BaseNTok}[1]{\textcolor[rgb]{0.00,0.00,0.81}{#1}}
\newcommand{\BuiltInTok}[1]{#1}
\newcommand{\CharTok}[1]{\textcolor[rgb]{0.31,0.60,0.02}{#1}}
\newcommand{\CommentTok}[1]{\textcolor[rgb]{0.56,0.35,0.01}{\textit{#1}}}
\newcommand{\CommentVarTok}[1]{\textcolor[rgb]{0.56,0.35,0.01}{\textbf{\textit{#1}}}}
\newcommand{\ConstantTok}[1]{\textcolor[rgb]{0.00,0.00,0.00}{#1}}
\newcommand{\ControlFlowTok}[1]{\textcolor[rgb]{0.13,0.29,0.53}{\textbf{#1}}}
\newcommand{\DataTypeTok}[1]{\textcolor[rgb]{0.13,0.29,0.53}{#1}}
\newcommand{\DecValTok}[1]{\textcolor[rgb]{0.00,0.00,0.81}{#1}}
\newcommand{\DocumentationTok}[1]{\textcolor[rgb]{0.56,0.35,0.01}{\textbf{\textit{#1}}}}
\newcommand{\ErrorTok}[1]{\textcolor[rgb]{0.64,0.00,0.00}{\textbf{#1}}}
\newcommand{\ExtensionTok}[1]{#1}
\newcommand{\FloatTok}[1]{\textcolor[rgb]{0.00,0.00,0.81}{#1}}
\newcommand{\FunctionTok}[1]{\textcolor[rgb]{0.00,0.00,0.00}{#1}}
\newcommand{\ImportTok}[1]{#1}
\newcommand{\InformationTok}[1]{\textcolor[rgb]{0.56,0.35,0.01}{\textbf{\textit{#1}}}}
\newcommand{\KeywordTok}[1]{\textcolor[rgb]{0.13,0.29,0.53}{\textbf{#1}}}
\newcommand{\NormalTok}[1]{#1}
\newcommand{\OperatorTok}[1]{\textcolor[rgb]{0.81,0.36,0.00}{\textbf{#1}}}
\newcommand{\OtherTok}[1]{\textcolor[rgb]{0.56,0.35,0.01}{#1}}
\newcommand{\PreprocessorTok}[1]{\textcolor[rgb]{0.56,0.35,0.01}{\textit{#1}}}
\newcommand{\RegionMarkerTok}[1]{#1}
\newcommand{\SpecialCharTok}[1]{\textcolor[rgb]{0.00,0.00,0.00}{#1}}
\newcommand{\SpecialStringTok}[1]{\textcolor[rgb]{0.31,0.60,0.02}{#1}}
\newcommand{\StringTok}[1]{\textcolor[rgb]{0.31,0.60,0.02}{#1}}
\newcommand{\VariableTok}[1]{\textcolor[rgb]{0.00,0.00,0.00}{#1}}
\newcommand{\VerbatimStringTok}[1]{\textcolor[rgb]{0.31,0.60,0.02}{#1}}
\newcommand{\WarningTok}[1]{\textcolor[rgb]{0.56,0.35,0.01}{\textbf{\textit{#1}}}}
\usepackage{graphicx}
\makeatletter
\def\maxwidth{\ifdim\Gin@nat@width>\linewidth\linewidth\else\Gin@nat@width\fi}
\def\maxheight{\ifdim\Gin@nat@height>\textheight\textheight\else\Gin@nat@height\fi}
\makeatother
% Scale images if necessary, so that they will not overflow the page
% margins by default, and it is still possible to overwrite the defaults
% using explicit options in \includegraphics[width, height, ...]{}
\setkeys{Gin}{width=\maxwidth,height=\maxheight,keepaspectratio}
% Set default figure placement to htbp
\makeatletter
\def\fps@figure{htbp}
\makeatother
\setlength{\emergencystretch}{3em} % prevent overfull lines
\providecommand{\tightlist}{%
  \setlength{\itemsep}{0pt}\setlength{\parskip}{0pt}}
\setcounter{secnumdepth}{-\maxdimen} % remove section numbering
\ifLuaTeX
  \usepackage{selnolig}  % disable illegal ligatures
\fi
\IfFileExists{bookmark.sty}{\usepackage{bookmark}}{\usepackage{hyperref}}
\IfFileExists{xurl.sty}{\usepackage{xurl}}{} % add URL line breaks if available
\urlstyle{same} % disable monospaced font for URLs
\hypersetup{
  pdftitle={M3: Exercise},
  pdfauthor={Luana Lima},
  hidelinks,
  pdfcreator={LaTeX via pandoc}}

\title{M3: Exercise}
\author{Luana Lima}
\date{}

\begin{document}
\maketitle

\hypertarget{setting-r-code-chunk-options}{%
\subsection{Setting R code chunk
options}\label{setting-r-code-chunk-options}}

First R code chunk is used for setting the options for all R code
chunks. The choice echo=TRUE means both code and output will appear on
report, include = FALSE neither code nor output is printed.

\hypertarget{loading-packages-and-initializing}{%
\subsection{Loading packages and
initializing}\label{loading-packages-and-initializing}}

Second R code chunk is for loading packages. By setting message = FALSE,
the code will appear but not the output.

\begin{Shaded}
\begin{Highlighting}[]
\FunctionTok{library}\NormalTok{(lubridate)}
\FunctionTok{library}\NormalTok{(ggplot2)}
\FunctionTok{library}\NormalTok{(forecast)}
\FunctionTok{library}\NormalTok{(Kendall)}
\FunctionTok{library}\NormalTok{(tseries)}
\end{Highlighting}
\end{Shaded}

\hypertarget{importing-data}{%
\subsection{Importing data}\label{importing-data}}

Let's continue working with our inflow data for reservoirs in Brazil.

\begin{Shaded}
\begin{Highlighting}[]
\CommentTok{\#Importing time series data from text file\#}
\NormalTok{raw\_inflow\_data }\OtherTok{\textless{}{-}} \FunctionTok{read.table}\NormalTok{(}\AttributeTok{file=}\StringTok{"../Data/inflowtimeseries.txt"}\NormalTok{,}\AttributeTok{header=}\ConstantTok{FALSE}\NormalTok{,}\AttributeTok{skip=}\DecValTok{0}\NormalTok{)}

\CommentTok{\#Trim the table to include only columns you need}
\NormalTok{nhydro }\OtherTok{\textless{}{-}} \FunctionTok{ncol}\NormalTok{(raw\_inflow\_data)}\SpecialCharTok{{-}}\DecValTok{2}
\NormalTok{nobs }\OtherTok{\textless{}{-}} \FunctionTok{nrow}\NormalTok{(raw\_inflow\_data) }

\CommentTok{\#If your file does not have header like this one you can add column names after }
\CommentTok{\#creating the data frame}
\FunctionTok{colnames}\NormalTok{(raw\_inflow\_data)}\OtherTok{=}\FunctionTok{c}\NormalTok{(}\StringTok{"Month"}\NormalTok{,}\StringTok{"Year"}\NormalTok{, }\StringTok{"HP1"}\NormalTok{, }\StringTok{"HP2"}\NormalTok{,}\StringTok{"HP3"}\NormalTok{,}\StringTok{"HP4"}\NormalTok{, }\StringTok{"HP5"}\NormalTok{,}
                            \StringTok{"HP6"}\NormalTok{,}\StringTok{"HP7"}\NormalTok{, }\StringTok{"HP8"}\NormalTok{,}\StringTok{"HP9"}\NormalTok{,}\StringTok{"HP10"}\NormalTok{, }\StringTok{"HP11"}\NormalTok{,}\StringTok{"HP12"}\NormalTok{, }
                            \StringTok{"HP13"}\NormalTok{, }\StringTok{"HP14"}\NormalTok{,}\StringTok{"HP15"}\NormalTok{)}

\CommentTok{\#Checking data}
\FunctionTok{head}\NormalTok{(raw\_inflow\_data)}
\end{Highlighting}
\end{Shaded}

\begin{verbatim}
##   Month Year  HP1  HP2  HP3  HP4  HP5  HP6 HP7  HP8 HP9 HP10 HP11 HP12 HP13
## 1   Jan 1931 4782 4076 2518 2450 2649 1462 450  968 246 2636  452 4870  452
## 2   Feb 1931 7323 7681 4188  150 2401  758 554  219  74 4158  457 4550  796
## 3   Mar 1931 8266 5921 3253 2389 3261  707 615  333 123 3847  631 6537  804
## 4   Apr 1931 6247 4600 2449 1253 2006  469 474  297 113 3291  510 7298  644
## 5   May 1931 3642 2789 1651 2374 2454 3167 378 3295 938 1956  276 4942  421
## 6   Jun 1931 2425 2062 1270 2672 2433 3236 301 2547 951 1371  201 2478  305
##    HP14  HP15
## 1 17342 31270
## 2 21530 43827
## 3 33299 49884
## 4 34674 43962
## 5 15184 35156
## 6  8611 25764
\end{verbatim}

\begin{Shaded}
\begin{Highlighting}[]
\FunctionTok{str}\NormalTok{(raw\_inflow\_data)}
\end{Highlighting}
\end{Shaded}

\begin{verbatim}
## 'data.frame':    972 obs. of  17 variables:
##  $ Month: chr  "Jan" "Feb" "Mar" "Apr" ...
##  $ Year : int  1931 1931 1931 1931 1931 1931 1931 1931 1931 1931 ...
##  $ HP1  : int  4782 7323 8266 6247 3642 2425 2158 1854 1839 1896 ...
##  $ HP2  : int  4076 7681 5921 4600 2789 2062 1644 1301 1439 1340 ...
##  $ HP3  : int  2518 4188 3253 2449 1651 1270 1204 1152 1297 1259 ...
##  $ HP4  : int  2450 150 2389 1253 2374 2672 1238 605 1016 674 ...
##  $ HP5  : int  2649 2401 3261 2006 2454 2433 1798 1160 1584 1563 ...
##  $ HP6  : int  1462 758 707 469 3167 3236 1957 844 1937 1484 ...
##  $ HP7  : int  450 554 615 474 378 301 256 244 222 355 ...
##  $ HP8  : int  968 219 333 297 3295 2547 2585 1173 3596 1140 ...
##  $ HP9  : int  246 74 123 113 938 951 883 404 378 211 ...
##  $ HP10 : int  2636 4158 3847 3291 1956 1371 1186 1049 1162 1507 ...
##  $ HP11 : int  452 457 631 510 276 201 213 196 161 208 ...
##  $ HP12 : int  4870 4550 6537 7298 4942 2478 1905 1647 1453 1358 ...
##  $ HP13 : int  452 796 804 644 421 305 261 246 250 328 ...
##  $ HP14 : int  17342 21530 33299 34674 15184 8611 5939 4259 3282 3305 ...
##  $ HP15 : int  31270 43827 49884 43962 35156 25764 18109 13320 8225 8900 ...
\end{verbatim}

\hypertarget{creating-the-date-object}{%
\subsection{Creating the date object}\label{creating-the-date-object}}

Here we use the function my() from package lubridate.

\begin{Shaded}
\begin{Highlighting}[]
\CommentTok{\#using package lubridate}
\NormalTok{my\_date }\OtherTok{\textless{}{-}} \FunctionTok{paste}\NormalTok{(raw\_inflow\_data[,}\DecValTok{1}\NormalTok{],raw\_inflow\_data[,}\DecValTok{2}\NormalTok{],}\AttributeTok{sep=}\StringTok{"{-}"}\NormalTok{)}
\NormalTok{my\_date }\OtherTok{\textless{}{-}} \FunctionTok{my}\NormalTok{(my\_date)  }\CommentTok{\#function my from package lubridate}
\FunctionTok{head}\NormalTok{(my\_date)}
\end{Highlighting}
\end{Shaded}

\begin{verbatim}
## [1] "1931-01-01" "1931-02-01" "1931-03-01" "1931-04-01" "1931-05-01"
## [6] "1931-06-01"
\end{verbatim}

\begin{Shaded}
\begin{Highlighting}[]
\CommentTok{\#add that to inflow\_data and store in a new data frame }
\NormalTok{inflow\_data }\OtherTok{\textless{}{-}} \FunctionTok{cbind}\NormalTok{(my\_date,raw\_inflow\_data[,}\DecValTok{3}\SpecialCharTok{:}\NormalTok{(}\DecValTok{3}\SpecialCharTok{+}\NormalTok{nhydro}\DecValTok{{-}1}\NormalTok{)])}
\FunctionTok{head}\NormalTok{(inflow\_data)}
\end{Highlighting}
\end{Shaded}

\begin{verbatim}
##      my_date  HP1  HP2  HP3  HP4  HP5  HP6 HP7  HP8 HP9 HP10 HP11 HP12 HP13
## 1 1931-01-01 4782 4076 2518 2450 2649 1462 450  968 246 2636  452 4870  452
## 2 1931-02-01 7323 7681 4188  150 2401  758 554  219  74 4158  457 4550  796
## 3 1931-03-01 8266 5921 3253 2389 3261  707 615  333 123 3847  631 6537  804
## 4 1931-04-01 6247 4600 2449 1253 2006  469 474  297 113 3291  510 7298  644
## 5 1931-05-01 3642 2789 1651 2374 2454 3167 378 3295 938 1956  276 4942  421
## 6 1931-06-01 2425 2062 1270 2672 2433 3236 301 2547 951 1371  201 2478  305
##    HP14  HP15
## 1 17342 31270
## 2 21530 43827
## 3 33299 49884
## 4 34674 43962
## 5 15184 35156
## 6  8611 25764
\end{verbatim}

\hypertarget{initial-plots}{%
\subsection{Initial Plots}\label{initial-plots}}

Initial time series plot.

\begin{Shaded}
\begin{Highlighting}[]
\CommentTok{\#using package ggplot2}
\ControlFlowTok{for}\NormalTok{(i }\ControlFlowTok{in} \DecValTok{1}\SpecialCharTok{:}\NormalTok{nhydro)\{}
  \FunctionTok{print}\NormalTok{(}\FunctionTok{ggplot}\NormalTok{(inflow\_data, }\FunctionTok{aes}\NormalTok{(}\AttributeTok{x=}\NormalTok{my\_date, }\AttributeTok{y=}\NormalTok{inflow\_data[,(}\DecValTok{1}\SpecialCharTok{+}\NormalTok{i)])) }\SpecialCharTok{+}
            \FunctionTok{geom\_line}\NormalTok{(}\AttributeTok{color=}\StringTok{"blue"}\NormalTok{) }\SpecialCharTok{+}
            \FunctionTok{ylab}\NormalTok{(}\FunctionTok{paste0}\NormalTok{(}\StringTok{"Inflow "}\NormalTok{,}\FunctionTok{colnames}\NormalTok{(inflow\_data)[(}\DecValTok{1}\SpecialCharTok{+}\NormalTok{i)],}\AttributeTok{sep=}\StringTok{""}\NormalTok{)) }
\NormalTok{        )}
\NormalTok{\}}
\end{Highlighting}
\end{Shaded}

\includegraphics{M3_TrendSeasonality_Complete_files/figure-latex/unnamed-chunk-3-1.pdf}
\includegraphics{M3_TrendSeasonality_Complete_files/figure-latex/unnamed-chunk-3-2.pdf}
\includegraphics{M3_TrendSeasonality_Complete_files/figure-latex/unnamed-chunk-3-3.pdf}
\includegraphics{M3_TrendSeasonality_Complete_files/figure-latex/unnamed-chunk-3-4.pdf}
\includegraphics{M3_TrendSeasonality_Complete_files/figure-latex/unnamed-chunk-3-5.pdf}
\includegraphics{M3_TrendSeasonality_Complete_files/figure-latex/unnamed-chunk-3-6.pdf}
\includegraphics{M3_TrendSeasonality_Complete_files/figure-latex/unnamed-chunk-3-7.pdf}
\includegraphics{M3_TrendSeasonality_Complete_files/figure-latex/unnamed-chunk-3-8.pdf}
\includegraphics{M3_TrendSeasonality_Complete_files/figure-latex/unnamed-chunk-3-9.pdf}
\includegraphics{M3_TrendSeasonality_Complete_files/figure-latex/unnamed-chunk-3-10.pdf}
\includegraphics{M3_TrendSeasonality_Complete_files/figure-latex/unnamed-chunk-3-11.pdf}
\includegraphics{M3_TrendSeasonality_Complete_files/figure-latex/unnamed-chunk-3-12.pdf}
\includegraphics{M3_TrendSeasonality_Complete_files/figure-latex/unnamed-chunk-3-13.pdf}
\includegraphics{M3_TrendSeasonality_Complete_files/figure-latex/unnamed-chunk-3-14.pdf}
\includegraphics{M3_TrendSeasonality_Complete_files/figure-latex/unnamed-chunk-3-15.pdf}

\hypertarget{zeros-in-the-end-on-data}{%
\subsection{Zeros in the end on data}\label{zeros-in-the-end-on-data}}

The initial plots showed that we have zeros in the end of the data set.
It could be missing observation or observation that haven't been
observed yet. Use the tail() to find out how many zeros you have and how
many lines you will need to remove.

\begin{Shaded}
\begin{Highlighting}[]
\CommentTok{\#check the final obs on data }
\FunctionTok{tail}\NormalTok{(inflow\_data)}
\end{Highlighting}
\end{Shaded}

\begin{verbatim}
##        my_date  HP1  HP2  HP3  HP4  HP5  HP6 HP7  HP8  HP9 HP10 HP11 HP12 HP13
## 967 2011-07-01 1883 1426 1560 2930 2105 2988 233 4578 2045  864  119 1068  275
## 968 2011-08-01 1444 1139 1441 5069 2328 4559 224 4573 2527  827  120  854  251
## 969 2011-09-01    0    0    0    0    0    0   0    0    0    0    0    0    0
## 970 2011-10-01    0    0    0    0    0    0   0    0    0    0    0    0    0
## 971 2011-11-01    0    0    0    0    0    0   0    0    0    0    0    0    0
## 972 2011-12-01    0    0    0    0    0    0   0    0    0    0    0    0    0
##     HP14  HP15
## 967 3910 14162
## 968 2561  8896
## 969    0     0
## 970    0     0
## 971    0     0
## 972    0     0
\end{verbatim}

Note our last observation is from August 2011 but the data file was
filled with zeros. Let's remove the last four rows of our data set.

\begin{Shaded}
\begin{Highlighting}[]
\CommentTok{\#Remove last for rows by replacing current data frame}
\NormalTok{inflow\_data }\OtherTok{\textless{}{-}}\NormalTok{ inflow\_data[}\DecValTok{1}\SpecialCharTok{:}\NormalTok{(nobs}\DecValTok{{-}12}\NormalTok{),]}

\CommentTok{\#update object with number of observations}
\NormalTok{nobs }\OtherTok{\textless{}{-}}\NormalTok{ nobs}\DecValTok{{-}12}

\CommentTok{\#Tail again to check if the rows were correctly removed}
\FunctionTok{tail}\NormalTok{(inflow\_data)}
\end{Highlighting}
\end{Shaded}

\begin{verbatim}
##        my_date  HP1  HP2  HP3  HP4  HP5  HP6 HP7  HP8  HP9 HP10 HP11 HP12 HP13
## 955 2010-07-01 1539 1214 1481 1978 1828 1449 227 2146 1161  932  124  867  255
## 956 2010-08-01 1289  886 1173 1490 1452 1238 233 1834  567  715  108  702  233
## 957 2010-09-01  953  798 1189  928 1564  439 212 1626 1219  645   58  610  216
## 958 2010-10-01 1411 1265 1580 2748 2268  971 251 1581  476  871  100  738  268
## 959 2010-11-01 2608 1681 1255 1721 1427  835 309 1109  415 1807  534 1726  336
## 960 2010-12-01 3338 2608 1921 3373 2203 3386 312 2908  453 3402  604 3064  380
##     HP14  HP15
## 955 2746 14043
## 956 1931  8815
## 957 1485  6512
## 958 1900  7492
## 959 3470 11387
## 960 7027 17839
\end{verbatim}

Fixed!

\hypertarget{transforming-data-into-time-series-object}{%
\subsection{Transforming data into time series
object}\label{transforming-data-into-time-series-object}}

Many of the functions we will use require a time series object. You can
transform your data in a time series using the function \emph{ts()}.

\begin{Shaded}
\begin{Highlighting}[]
\NormalTok{ts\_inflow\_data }\OtherTok{\textless{}{-}} \FunctionTok{ts}\NormalTok{(inflow\_data[,}\DecValTok{2}\SpecialCharTok{:}\NormalTok{(}\DecValTok{2}\SpecialCharTok{+}\NormalTok{nhydro}\DecValTok{{-}1}\NormalTok{)],}\AttributeTok{frequency=}\DecValTok{12}\NormalTok{) }
\CommentTok{\#note that we are only transforming columns with inflow data, not the date columns  \#start=my\_date[1],end=my\_date[nobs],}
\FunctionTok{head}\NormalTok{(ts\_inflow\_data,}\DecValTok{15}\NormalTok{)}
\end{Highlighting}
\end{Shaded}

\begin{verbatim}
##        HP1  HP2  HP3  HP4  HP5  HP6 HP7  HP8 HP9 HP10 HP11 HP12 HP13  HP14
## Jan 1 4782 4076 2518 2450 2649 1462 450  968 246 2636  452 4870  452 17342
## Feb 1 7323 7681 4188  150 2401  758 554  219  74 4158  457 4550  796 21530
## Mar 1 8266 5921 3253 2389 3261  707 615  333 123 3847  631 6537  804 33299
## Apr 1 6247 4600 2449 1253 2006  469 474  297 113 3291  510 7298  644 34674
## May 1 3642 2789 1651 2374 2454 3167 378 3295 938 1956  276 4942  421 15184
## Jun 1 2425 2062 1270 2672 2433 3236 301 2547 951 1371  201 2478  305  8611
## Jul 1 2158 1644 1204 1238 1798 1957 256 2585 883 1186  213 1905  261  5939
## Aug 1 1854 1301 1152  605 1160  844 244 1173 404 1049  196 1647  246  4259
## Sep 1 1839 1439 1297 1016 1584 1937 222 3596 378 1162  161 1453  250  3282
## Oct 1 1896 1340 1259  674 1563 1484 355 1140 211 1507  208 1358  328  3305
## Nov 1 2095 1447 1218  674 1404  835 371  563 252 1996  596 1905  319  6500
## Dec 1 2725 2479 2013 1278 2272 1073 419  512 197 3015  381 2121  335  8461
## Jan 2 4679 4021 2435 1259 1995 1044 520  609 159 3978  711 3811  467 14002
## Feb 2 5535 4082 2262 1895 2996 1454 525 1219 268 2615  316 4681  531 20596
## Mar 2 4310 3398 2065 1686 2392 1888 674 1332 304 2269  271 3329  501 21638
##        HP15
## Jan 1 31270
## Feb 1 43827
## Mar 1 49884
## Apr 1 43962
## May 1 35156
## Jun 1 25764
## Jul 1 18109
## Aug 1 13320
## Sep 1  8225
## Oct 1  8900
## Nov 1 13766
## Dec 1 20880
## Jan 2 33160
## Feb 2 39791
## Mar 2 48274
\end{verbatim}

Note that ts\_inflow\_data has information on start, end and frequency.

\hypertarget{plotting-acf-and-pacf}{%
\subsection{Plotting ACF and PACF}\label{plotting-acf-and-pacf}}

Let's use functions Acf() and Pacf() from package ``forecast''.

\begin{Shaded}
\begin{Highlighting}[]
\CommentTok{\#Acf and Pacf for HP1}
\ControlFlowTok{for}\NormalTok{(i }\ControlFlowTok{in} \DecValTok{1}\SpecialCharTok{:}\NormalTok{nhydro)\{}
  \FunctionTok{par}\NormalTok{(}\AttributeTok{mfrow=}\FunctionTok{c}\NormalTok{(}\DecValTok{1}\NormalTok{,}\DecValTok{2}\NormalTok{))  }\CommentTok{\#place plot side by side}
  \FunctionTok{Acf}\NormalTok{(ts\_inflow\_data[,i],}\AttributeTok{lag.max=}\DecValTok{40}\NormalTok{,}\AttributeTok{main=}\FunctionTok{paste}\NormalTok{(}\StringTok{"Inflows HP"}\NormalTok{,i,}\AttributeTok{sep=}\StringTok{""}\NormalTok{)) }
  \CommentTok{\# because I am not storing Acf() into any object, I don\textquotesingle{}t need to specify plot=TRUE }
  \FunctionTok{Pacf}\NormalTok{(ts\_inflow\_data[,i],}\AttributeTok{lag.max=}\DecValTok{40}\NormalTok{,}\AttributeTok{main=}\FunctionTok{paste}\NormalTok{(}\StringTok{"Inflows HP"}\NormalTok{,i,}\AttributeTok{sep=}\StringTok{""}\NormalTok{))}
\NormalTok{\}}
\end{Highlighting}
\end{Shaded}

\includegraphics{M3_TrendSeasonality_Complete_files/figure-latex/acf-1.pdf}
\includegraphics{M3_TrendSeasonality_Complete_files/figure-latex/acf-2.pdf}
\includegraphics{M3_TrendSeasonality_Complete_files/figure-latex/acf-3.pdf}
\includegraphics{M3_TrendSeasonality_Complete_files/figure-latex/acf-4.pdf}
\includegraphics{M3_TrendSeasonality_Complete_files/figure-latex/acf-5.pdf}
\includegraphics{M3_TrendSeasonality_Complete_files/figure-latex/acf-6.pdf}
\includegraphics{M3_TrendSeasonality_Complete_files/figure-latex/acf-7.pdf}
\includegraphics{M3_TrendSeasonality_Complete_files/figure-latex/acf-8.pdf}
\includegraphics{M3_TrendSeasonality_Complete_files/figure-latex/acf-9.pdf}
\includegraphics{M3_TrendSeasonality_Complete_files/figure-latex/acf-10.pdf}
\includegraphics{M3_TrendSeasonality_Complete_files/figure-latex/acf-11.pdf}
\includegraphics{M3_TrendSeasonality_Complete_files/figure-latex/acf-12.pdf}
\includegraphics{M3_TrendSeasonality_Complete_files/figure-latex/acf-13.pdf}
\includegraphics{M3_TrendSeasonality_Complete_files/figure-latex/acf-14.pdf}
\includegraphics{M3_TrendSeasonality_Complete_files/figure-latex/acf-15.pdf}

\hypertarget{trend-component}{%
\subsection{Trend Component}\label{trend-component}}

Let's identify and remove trend component like we leaned on the recorded
videos for M4. You start by fitting a linear model to
\(Y_t=\beta_0+\beta_1*t+\epsilon_t\).

\begin{Shaded}
\begin{Highlighting}[]
\CommentTok{\#Create vector t}
\NormalTok{t }\OtherTok{\textless{}{-}} \FunctionTok{c}\NormalTok{(}\DecValTok{1}\SpecialCharTok{:}\NormalTok{nobs)}

\CommentTok{\#Choose one hydro plant to study, as an exercise try to generalize this routine for all 15 HP}
\CommentTok{\#from the plot HP4 seems to have a trend so let\textquotesingle{}s play with that column}
\NormalTok{iHP}\OtherTok{=}\DecValTok{4}  \CommentTok{\#change this to chekc other HP}
\CommentTok{\#prep\_data \textless{}{-} data.frame("Inflow"=inflow\_data[,iHP],"Time"=t)}

\CommentTok{\#Fit a linear trend to TS of iHP}
\NormalTok{linear\_trend\_model}\OtherTok{=}\FunctionTok{lm}\NormalTok{(inflow\_data[,iHP}\SpecialCharTok{+}\DecValTok{1}\NormalTok{]}\SpecialCharTok{\textasciitilde{}}\NormalTok{t)  }
\FunctionTok{summary}\NormalTok{(linear\_trend\_model)}
\end{Highlighting}
\end{Shaded}

\begin{verbatim}
## 
## Call:
## lm(formula = inflow_data[, iHP + 1] ~ t)
## 
## Residuals:
##     Min      1Q  Median      3Q     Max 
## -2057.2  -691.1  -217.4   503.6  5786.1 
## 
## Coefficients:
##             Estimate Std. Error t value Pr(>|t|)    
## (Intercept) 637.5519    65.4250   9.745   <2e-16 ***
## t             2.1836     0.1179  18.513   <2e-16 ***
## ---
## Signif. codes:  0 '***' 0.001 '**' 0.01 '*' 0.05 '.' 0.1 ' ' 1
## 
## Residual standard error: 1013 on 958 degrees of freedom
## Multiple R-squared:  0.2635, Adjusted R-squared:  0.2627 
## F-statistic: 342.7 on 1 and 958 DF,  p-value: < 2.2e-16
\end{verbatim}

\begin{Shaded}
\begin{Highlighting}[]
\NormalTok{beta0}\OtherTok{=}\FunctionTok{as.numeric}\NormalTok{(linear\_trend\_model}\SpecialCharTok{$}\NormalTok{coefficients[}\DecValTok{1}\NormalTok{])  }\CommentTok{\#first coefficient is the intercept term or beta0}
\NormalTok{beta1}\OtherTok{=}\FunctionTok{as.numeric}\NormalTok{(linear\_trend\_model}\SpecialCharTok{$}\NormalTok{coefficients[}\DecValTok{2}\NormalTok{])  }\CommentTok{\#second coefficient is the slope or beta1}

\CommentTok{\#Let\textquotesingle{}s plot the time series with its trend line}
\FunctionTok{ggplot}\NormalTok{(inflow\_data, }\FunctionTok{aes}\NormalTok{(}\AttributeTok{x=}\NormalTok{my\_date, }\AttributeTok{y=}\NormalTok{inflow\_data[,(}\DecValTok{1}\SpecialCharTok{+}\NormalTok{iHP)])) }\SpecialCharTok{+}
            \FunctionTok{geom\_line}\NormalTok{(}\AttributeTok{color=}\StringTok{"blue"}\NormalTok{) }\SpecialCharTok{+}
            \FunctionTok{ylab}\NormalTok{(}\FunctionTok{paste0}\NormalTok{(}\StringTok{"Inflow "}\NormalTok{,}\FunctionTok{colnames}\NormalTok{(inflow\_data)[(}\DecValTok{1}\SpecialCharTok{+}\NormalTok{iHP)],}\AttributeTok{sep=}\StringTok{""}\NormalTok{)) }\SpecialCharTok{+}
            \CommentTok{\#geom\_abline(intercept = beta0, slope = beta1, color="red")}
            \FunctionTok{geom\_smooth}\NormalTok{(}\AttributeTok{color=}\StringTok{"red"}\NormalTok{,}\AttributeTok{method=}\StringTok{"lm"}\NormalTok{) }
\end{Highlighting}
\end{Shaded}

\begin{verbatim}
## `geom_smooth()` using formula = 'y ~ x'
\end{verbatim}

\includegraphics{M3_TrendSeasonality_Complete_files/figure-latex/unnamed-chunk-7-1.pdf}

\begin{Shaded}
\begin{Highlighting}[]
\CommentTok{\#remove the trend from series}
\NormalTok{detrend\_inflow\_data }\OtherTok{\textless{}{-}}\NormalTok{ inflow\_data[,(iHP}\SpecialCharTok{+}\DecValTok{1}\NormalTok{)]}\SpecialCharTok{{-}}\NormalTok{(beta0}\SpecialCharTok{+}\NormalTok{beta1}\SpecialCharTok{*}\NormalTok{t)}

\CommentTok{\#Understanding what we did}
\FunctionTok{ggplot}\NormalTok{(inflow\_data, }\FunctionTok{aes}\NormalTok{(}\AttributeTok{x=}\NormalTok{my\_date, }\AttributeTok{y=}\NormalTok{inflow\_data[,(}\DecValTok{1}\SpecialCharTok{+}\NormalTok{iHP)])) }\SpecialCharTok{+}
            \FunctionTok{geom\_line}\NormalTok{(}\AttributeTok{color=}\StringTok{"blue"}\NormalTok{) }\SpecialCharTok{+}
            \FunctionTok{ylab}\NormalTok{(}\FunctionTok{paste0}\NormalTok{(}\StringTok{"Inflow "}\NormalTok{,}\FunctionTok{colnames}\NormalTok{(inflow\_data)[(}\DecValTok{1}\SpecialCharTok{+}\NormalTok{iHP)],}\AttributeTok{sep=}\StringTok{""}\NormalTok{)) }\SpecialCharTok{+}
            \CommentTok{\#geom\_abline(intercept = beta0, slope = beta1, color="red")}
            \FunctionTok{geom\_smooth}\NormalTok{(}\AttributeTok{color=}\StringTok{"red"}\NormalTok{,}\AttributeTok{method=}\StringTok{"lm"}\NormalTok{) }\SpecialCharTok{+}
            \FunctionTok{geom\_line}\NormalTok{(}\FunctionTok{aes}\NormalTok{(}\AttributeTok{y=}\NormalTok{detrend\_inflow\_data), }\AttributeTok{col=}\StringTok{"green"}\NormalTok{)}\SpecialCharTok{+}
            \FunctionTok{geom\_smooth}\NormalTok{(}\FunctionTok{aes}\NormalTok{(}\AttributeTok{y=}\NormalTok{detrend\_inflow\_data),}\AttributeTok{color=}\StringTok{"orange"}\NormalTok{,}\AttributeTok{method=}\StringTok{"lm"}\NormalTok{) }
\end{Highlighting}
\end{Shaded}

\begin{verbatim}
## `geom_smooth()` using formula = 'y ~ x'
## `geom_smooth()` using formula = 'y ~ x'
\end{verbatim}

\includegraphics{M3_TrendSeasonality_Complete_files/figure-latex/unnamed-chunk-8-1.pdf}
Note that blue line is our original series, red lien is our trend, green
line is our original series minus the trend or in other words the
detrended series. And in orange is the trend line for the detrended
series which has slope 0 meaning we were able to effectively eliminate
the trend with a linear model.

\hypertarget{seasonal-component}{%
\subsection{Seasonal Component}\label{seasonal-component}}

Now let's shift attention to the seasonal component.

\begin{Shaded}
\begin{Highlighting}[]
\CommentTok{\#Let\textquotesingle{}s choose another HP}
\NormalTok{iHP}\OtherTok{=}\DecValTok{1} 

\CommentTok{\#Use seasonal means model}
\CommentTok{\#First create the seasonal dummies}
\NormalTok{dummies }\OtherTok{\textless{}{-}} \FunctionTok{seasonaldummy}\NormalTok{(ts\_inflow\_data[,iHP])  }
\CommentTok{\#this function only accepts ts object, no need to add one here because date }
\CommentTok{\#object is not a column}

\CommentTok{\#Then fit a linear model to the seasonal dummies}
\NormalTok{seas\_means\_model}\OtherTok{=}\FunctionTok{lm}\NormalTok{(inflow\_data[,(iHP}\SpecialCharTok{+}\DecValTok{1}\NormalTok{)]}\SpecialCharTok{\textasciitilde{}}\NormalTok{dummies)}
\FunctionTok{summary}\NormalTok{(seas\_means\_model)}
\end{Highlighting}
\end{Shaded}

\begin{verbatim}
## 
## Call:
## lm(formula = inflow_data[, (iHP + 1)] ~ dummies)
## 
## Residuals:
##     Min      1Q  Median      3Q     Max 
## -3414.4  -451.6   -45.7   336.6  5962.6 
## 
## Coefficients:
##             Estimate Std. Error t value Pr(>|t|)    
## (Intercept)   3482.7      114.5  30.418  < 2e-16 ***
## dummiesJan    1202.3      161.9   7.425 2.51e-13 ***
## dummiesFeb    1469.7      161.9   9.076  < 2e-16 ***
## dummiesMar    1377.8      161.9   8.509  < 2e-16 ***
## dummiesApr     241.8      161.9   1.493    0.136    
## dummiesMay    -999.5      161.9  -6.172 9.96e-10 ***
## dummiesJun   -1529.7      161.9  -9.447  < 2e-16 ***
## dummiesJul   -1887.2      161.9 -11.655  < 2e-16 ***
## dummiesAug   -2155.7      161.9 -13.313  < 2e-16 ***
## dummiesSep   -2245.4      161.9 -13.867  < 2e-16 ***
## dummiesOct   -2018.1      161.9 -12.464  < 2e-16 ***
## dummiesNov   -1335.4      161.9  -8.247 5.39e-16 ***
## ---
## Signif. codes:  0 '***' 0.001 '**' 0.01 '*' 0.05 '.' 0.1 ' ' 1
## 
## Residual standard error: 1024 on 948 degrees of freedom
## Multiple R-squared:  0.6476, Adjusted R-squared:  0.6435 
## F-statistic: 158.3 on 11 and 948 DF,  p-value: < 2.2e-16
\end{verbatim}

\begin{Shaded}
\begin{Highlighting}[]
\CommentTok{\#Look at the regression coefficient. These will be the values of Beta}

\CommentTok{\#Store regression coefficients}
\NormalTok{beta\_int}\OtherTok{=}\NormalTok{seas\_means\_model}\SpecialCharTok{$}\NormalTok{coefficients[}\DecValTok{1}\NormalTok{]}
\NormalTok{beta\_coeff}\OtherTok{=}\NormalTok{seas\_means\_model}\SpecialCharTok{$}\NormalTok{coefficients[}\DecValTok{2}\SpecialCharTok{:}\DecValTok{12}\NormalTok{]}

\CommentTok{\#compute seasonal component}
\NormalTok{inflow\_seas\_comp}\OtherTok{=}\FunctionTok{array}\NormalTok{(}\DecValTok{0}\NormalTok{,nobs)}
\ControlFlowTok{for}\NormalTok{(i }\ControlFlowTok{in} \DecValTok{1}\SpecialCharTok{:}\NormalTok{nobs)\{}
\NormalTok{  inflow\_seas\_comp[i]}\OtherTok{=}\NormalTok{(beta\_int}\SpecialCharTok{+}\NormalTok{beta\_coeff}\SpecialCharTok{\%*\%}\NormalTok{dummies[i,])}
\NormalTok{\}}

\CommentTok{\#Understanding what we did}
\FunctionTok{ggplot}\NormalTok{(inflow\_data, }\FunctionTok{aes}\NormalTok{(}\AttributeTok{x=}\NormalTok{my\_date, }\AttributeTok{y=}\NormalTok{inflow\_data[,(}\DecValTok{1}\SpecialCharTok{+}\NormalTok{iHP)])) }\SpecialCharTok{+}
            \FunctionTok{geom\_line}\NormalTok{(}\AttributeTok{color=}\StringTok{"blue"}\NormalTok{) }\SpecialCharTok{+}
            \FunctionTok{ylab}\NormalTok{(}\FunctionTok{paste0}\NormalTok{(}\StringTok{"Inflow "}\NormalTok{,}\FunctionTok{colnames}\NormalTok{(inflow\_data)[(}\DecValTok{1}\SpecialCharTok{+}\NormalTok{iHP)],}\AttributeTok{sep=}\StringTok{""}\NormalTok{)) }\SpecialCharTok{+}
            \FunctionTok{geom\_line}\NormalTok{(}\FunctionTok{aes}\NormalTok{(}\AttributeTok{y=}\NormalTok{inflow\_seas\_comp), }\AttributeTok{col=}\StringTok{"red"}\NormalTok{)}
\end{Highlighting}
\end{Shaded}

\includegraphics{M3_TrendSeasonality_Complete_files/figure-latex/unnamed-chunk-9-1.pdf}

\begin{Shaded}
\begin{Highlighting}[]
\CommentTok{\#Removing seasonal component}
\NormalTok{deseason\_inflow\_data }\OtherTok{\textless{}{-}}\NormalTok{ inflow\_data[,(}\DecValTok{1}\SpecialCharTok{+}\NormalTok{iHP)]}\SpecialCharTok{{-}}\NormalTok{inflow\_seas\_comp}

\CommentTok{\#Understanding what we did}
\FunctionTok{ggplot}\NormalTok{(inflow\_data, }\FunctionTok{aes}\NormalTok{(}\AttributeTok{x=}\NormalTok{my\_date, }\AttributeTok{y=}\NormalTok{inflow\_data[,(}\DecValTok{1}\SpecialCharTok{+}\NormalTok{iHP)])) }\SpecialCharTok{+}
            \FunctionTok{geom\_line}\NormalTok{(}\AttributeTok{color=}\StringTok{"blue"}\NormalTok{) }\SpecialCharTok{+}
            \FunctionTok{ylab}\NormalTok{(}\FunctionTok{paste0}\NormalTok{(}\StringTok{"Inflow "}\NormalTok{,}\FunctionTok{colnames}\NormalTok{(inflow\_data)[(}\DecValTok{1}\SpecialCharTok{+}\NormalTok{iHP)],}\AttributeTok{sep=}\StringTok{""}\NormalTok{)) }\SpecialCharTok{+}
            \FunctionTok{geom\_line}\NormalTok{(}\FunctionTok{aes}\NormalTok{(}\AttributeTok{y=}\NormalTok{deseason\_inflow\_data), }\AttributeTok{col=}\StringTok{"green"}\NormalTok{)}
\end{Highlighting}
\end{Shaded}

\includegraphics{M3_TrendSeasonality_Complete_files/figure-latex/unnamed-chunk-9-2.pdf}

\hypertarget{stationarity-tests-in-r}{%
\subsection{Stationarity Tests in R}\label{stationarity-tests-in-r}}

Before we get started with the tests recall that for the year 2011 we
only have data until Aug 2011. Let's filter our dataset to remove that
last year.

\begin{Shaded}
\begin{Highlighting}[]
\NormalTok{ts\_inflow\_data }\OtherTok{\textless{}{-}} \FunctionTok{as.ts}\NormalTok{(ts\_inflow\_data[}\DecValTok{1}\SpecialCharTok{:}\DecValTok{960}\NormalTok{,])}
\NormalTok{my\_date }\OtherTok{\textless{}{-}}\NormalTok{ my\_date[}\DecValTok{1}\SpecialCharTok{:}\DecValTok{960}\NormalTok{]}

\CommentTok{\#create new df}
\NormalTok{inflow\_data\_new }\OtherTok{\textless{}{-}} \FunctionTok{cbind}\NormalTok{(my\_date, ts\_inflow\_data)}

\CommentTok{\#note that there are more elegant ways to do that using filter function from package dplyr}
\CommentTok{\#Let me know if you would like to learn how to use them}
\end{Highlighting}
\end{Shaded}

Some test only work for non-seasonal data. SO let's create another
series with yearly averages for inflow.

\begin{Shaded}
\begin{Highlighting}[]
\NormalTok{iHP}\OtherTok{=}\DecValTok{1}
\CommentTok{\#Group data in yearly steps instances}
\NormalTok{inflow\_data\_matrix }\OtherTok{\textless{}{-}} \FunctionTok{matrix}\NormalTok{(ts\_inflow\_data[,iHP],}\AttributeTok{byrow=}\ConstantTok{FALSE}\NormalTok{,}\AttributeTok{nrow=}\DecValTok{12}\NormalTok{)}
\NormalTok{inflow\_data\_yearly }\OtherTok{\textless{}{-}} \FunctionTok{colMeans}\NormalTok{(inflow\_data\_matrix)}

\FunctionTok{library}\NormalTok{(dplyr)  }\CommentTok{\#move this to package chunk later}
\end{Highlighting}
\end{Shaded}

\begin{verbatim}
## 
## 载入程辑包:'dplyr'
\end{verbatim}

\begin{verbatim}
## The following objects are masked from 'package:stats':
## 
##     filter, lag
\end{verbatim}

\begin{verbatim}
## The following objects are masked from 'package:base':
## 
##     intersect, setdiff, setequal, union
\end{verbatim}

\begin{Shaded}
\begin{Highlighting}[]
\NormalTok{my\_year }\OtherTok{\textless{}{-}} \FunctionTok{c}\NormalTok{(}\FunctionTok{year}\NormalTok{(}\FunctionTok{first}\NormalTok{(my\_date))}\SpecialCharTok{:}\FunctionTok{year}\NormalTok{(}\FunctionTok{last}\NormalTok{(my\_date)))}

\NormalTok{inflow\_data\_new\_yearly }\OtherTok{\textless{}{-}} \FunctionTok{data.frame}\NormalTok{(my\_year, inflow\_data\_yearly)}

\FunctionTok{ggplot}\NormalTok{(inflow\_data\_new\_yearly, }\FunctionTok{aes}\NormalTok{(}\AttributeTok{x=}\NormalTok{my\_year, }\AttributeTok{y=}\NormalTok{inflow\_data\_yearly)) }\SpecialCharTok{+}
            \FunctionTok{geom\_line}\NormalTok{(}\AttributeTok{color=}\StringTok{"blue"}\NormalTok{) }\SpecialCharTok{+}
            \FunctionTok{geom\_smooth}\NormalTok{(}\AttributeTok{color=}\StringTok{"red"}\NormalTok{,}\AttributeTok{method=}\StringTok{"lm"}\NormalTok{) }
\end{Highlighting}
\end{Shaded}

\begin{verbatim}
## `geom_smooth()` using formula = 'y ~ x'
\end{verbatim}

\includegraphics{M3_TrendSeasonality_Complete_files/figure-latex/unnamed-chunk-11-1.pdf}

\hypertarget{mann-kendall}{%
\subsubsection{Mann Kendall}\label{mann-kendall}}

Check for deterministic trend.

\begin{Shaded}
\begin{Highlighting}[]
\CommentTok{\#Since I have seasonal data I cannot use the simple MannKendall()}
\CommentTok{\#another example of functions that need a ts object}
\NormalTok{SMKtest }\OtherTok{\textless{}{-}} \FunctionTok{SeasonalMannKendall}\NormalTok{(ts\_inflow\_data[,iHP])}
\FunctionTok{print}\NormalTok{(}\StringTok{"Results for Seasonal Mann Kendall /n"}\NormalTok{)}
\end{Highlighting}
\end{Shaded}

\begin{verbatim}
## [1] "Results for Seasonal Mann Kendall /n"
\end{verbatim}

\begin{Shaded}
\begin{Highlighting}[]
\FunctionTok{print}\NormalTok{(}\FunctionTok{summary}\NormalTok{(SMKtest))}
\end{Highlighting}
\end{Shaded}

\begin{verbatim}
## Score =  37242 , Var(Score) = 98457224
## denominator =  460269
## tau = 0.0809, 2-sided pvalue =0.00017455
## NULL
\end{verbatim}

\begin{Shaded}
\begin{Highlighting}[]
\CommentTok{\#Use yearly date to run Mann Kendall}
\FunctionTok{print}\NormalTok{(}\StringTok{"Results of Mann Kendall on average yearly series"}\NormalTok{)}
\end{Highlighting}
\end{Shaded}

\begin{verbatim}
## [1] "Results of Mann Kendall on average yearly series"
\end{verbatim}

\begin{Shaded}
\begin{Highlighting}[]
\FunctionTok{print}\NormalTok{(}\FunctionTok{summary}\NormalTok{(}\FunctionTok{MannKendall}\NormalTok{(inflow\_data\_yearly)))}
\end{Highlighting}
\end{Shaded}

\begin{verbatim}
## Score =  618 , Var(Score) = 57933.33
## denominator =  3160
## tau = 0.196, 2-sided pvalue =0.010364
## NULL
\end{verbatim}

\hypertarget{spearman-correlation-test}{%
\subsubsection{Spearman Correlation
test}\label{spearman-correlation-test}}

Also check for deterministic trend, for non-seasonal data only.

\begin{Shaded}
\begin{Highlighting}[]
\CommentTok{\#Deterministic trend with Spearman Correlation Test}
\FunctionTok{print}\NormalTok{(}\StringTok{"Results from Spearman Correlation"}\NormalTok{)}
\end{Highlighting}
\end{Shaded}

\begin{verbatim}
## [1] "Results from Spearman Correlation"
\end{verbatim}

\begin{Shaded}
\begin{Highlighting}[]
\NormalTok{sp\_rho}\OtherTok{=}\FunctionTok{cor}\NormalTok{(inflow\_data\_yearly,my\_year,}\AttributeTok{method=}\StringTok{"spearman"}\NormalTok{)}
\FunctionTok{print}\NormalTok{(sp\_rho)}
\end{Highlighting}
\end{Shaded}

\begin{verbatim}
## [1] 0.2847398
\end{verbatim}

\begin{Shaded}
\begin{Highlighting}[]
\CommentTok{\#with cor.test you can get test statistics}
\NormalTok{sp\_rho}\OtherTok{=}\FunctionTok{cor.test}\NormalTok{(inflow\_data\_yearly,my\_year,}\AttributeTok{method=}\StringTok{"spearman"}\NormalTok{)}
\FunctionTok{print}\NormalTok{(sp\_rho)}
\end{Highlighting}
\end{Shaded}

\begin{verbatim}
## 
##  Spearman's rank correlation rho
## 
## data:  inflow_data_yearly and my_year
## S = 61026, p-value = 0.01069
## alternative hypothesis: true rho is not equal to 0
## sample estimates:
##       rho 
## 0.2847398
\end{verbatim}

\hypertarget{augmented-dickey-fuller-adf}{%
\subsubsection{Augmented Dickey Fuller
(ADF)}\label{augmented-dickey-fuller-adf}}

Used to detect stochastic trend.

\begin{Shaded}
\begin{Highlighting}[]
\CommentTok{\#Null hypothesis is that data has a unit root}
\FunctionTok{print}\NormalTok{(}\StringTok{"Results for ADF test/n"}\NormalTok{)}
\end{Highlighting}
\end{Shaded}

\begin{verbatim}
## [1] "Results for ADF test/n"
\end{verbatim}

\begin{Shaded}
\begin{Highlighting}[]
\FunctionTok{print}\NormalTok{(}\FunctionTok{adf.test}\NormalTok{(ts\_inflow\_data[,iHP],}\AttributeTok{alternative =} \StringTok{"stationary"}\NormalTok{))}
\end{Highlighting}
\end{Shaded}

\begin{verbatim}
## Warning in adf.test(ts_inflow_data[, iHP], alternative = "stationary"): p-value
## smaller than printed p-value
\end{verbatim}

\begin{verbatim}
## 
##  Augmented Dickey-Fuller Test
## 
## data:  ts_inflow_data[, iHP]
## Dickey-Fuller = -7.07, Lag order = 9, p-value = 0.01
## alternative hypothesis: stationary
\end{verbatim}

\begin{Shaded}
\begin{Highlighting}[]
\CommentTok{\#Now let\textquotesingle{}s try the yearly data}
\FunctionTok{print}\NormalTok{(}\StringTok{"Results for ADF test on yearly data/n"}\NormalTok{)}
\end{Highlighting}
\end{Shaded}

\begin{verbatim}
## [1] "Results for ADF test on yearly data/n"
\end{verbatim}

\begin{Shaded}
\begin{Highlighting}[]
\FunctionTok{print}\NormalTok{(}\FunctionTok{adf.test}\NormalTok{(inflow\_data\_yearly, }\AttributeTok{alternative =} \StringTok{"stationary"}\NormalTok{))}
\end{Highlighting}
\end{Shaded}

\begin{verbatim}
## 
##  Augmented Dickey-Fuller Test
## 
## data:  inflow_data_yearly
## Dickey-Fuller = -3.5308, Lag order = 4, p-value = 0.04481
## alternative hypothesis: stationary
\end{verbatim}

\#\#Exercise

Run the stationarity tests for the other variables HP2, HP3, \ldots{}

\end{document}
